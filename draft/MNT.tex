%&pdflatex
\documentclass[12pt]{article}

\usepackage{url,enumerate, amssymb, anysize, booktabs, amsfonts}
\usepackage[colorlinks = true,
linkcolor = blue,
urlcolor  = blue,
citecolor = green,
anchorcolor = blue]{hyperref}
\usepackage{setspace,listings}
\usepackage[dvipdfmx]{graphicx}
\usepackage{amsmath}
\usepackage{psfrag}
\usepackage{enumerate}
\usepackage{natbib}
\usepackage{url} % not crucial - just used below for the URL 
%\usepackage[style=chicago-notes-df]{biblatex}
%\pdfminorversion=4
% NOTE: To produce blinded version, replace "0" with "1" below.
\newcommand{\blind}{1}

% DON'T change margins - should be 1 inch all around.
\addtolength{\oddsidemargin}{-.5in}%
\addtolength{\evensidemargin}{-.5in}%
\addtolength{\textwidth}{1in}%
\addtolength{\textheight}{-.3in}%
\addtolength{\topmargin}{-.8in}%


\newcommand{\pb}{\mathbb{P}}
\newcommand{\E}{\ensuremath{\mbox{E}}}


\usepackage{color,amssymb}
\usepackage{fancyhdr, mathtools}
\usepackage{dcolumn}
\usepackage{indentfirst, verbatim}
\newcounter{equationset, sectsty, breqn}
\usepackage{setspace,float,lscape,amsmath,color}
\usepackage{color,amssymb}
\usepackage{fancyhdr, mathtools, amsthm}
\theoremstyle{definition}
\newtheorem{definition}{Definition}[section]
\newtheorem{theorem}{Theorem}[section]
\newtheorem{corollary}{Corollary}[theorem]
\newtheorem{lemma}[theorem]{Lemma}
\newtheorem{remark}{Remark}

\begin{document}
	
	%\bibliographystyle{natbib}
	
	\def\spacingset#1{\renewcommand{\baselinestretch}%
		{#1}\small\normalsize} \spacingset{1}
	
	
	%%%%%%%%%%%%%%%%%%%%%%%%%%%%%%%%%%%%%%%%%%%%%%%%%%%%%%%%%%%%%%%%%%%%%%%%%%%%%%
	
	\if1\blind
	{
		\title{\bf Testing independence between networks and nodal attributes via multiscale metrics}
		\author{Youjin Lee\thanks{
				The authors gratefully acknowledge \textit{please remember to list all relevant funding sources in the unblinded version}}\hspace{.2cm}\\
			Department of Biostatistics, Johns Hopkins School of Public Health\\
			and \\
			Author 2 \\
			Department of ZZZ, University of WWW}
		\maketitle
	} \fi
	
	\if0\blind
	{
		\bigskip
		\bigskip
		\bigskip
		\begin{center}
			{\LARGE\bf Testing independence between networks and nodal attributes via multiscale metrics}
		\end{center}
		\medskip
	} \fi
	
	
	\sloppy
	\bigskip
	\begin{abstract}
		%The text of your abstract. 200 or fewer words.
		Network dependence over network space, which refers to the dependence between network topology and its nodal attributes, often exhibits nonlinear dependent properties. Unfortunately, without knowledge on specific neighborhood structures, no statistic has been suggested to test network dependence further than globally linear dependence. This paper introduces a multiscale dependence test statistic called Multiscale Network Test (\texttt{MNT}), which borrows the idea of diffusion maps and Multiscale Generalized Correlation (\texttt{MGC}). Our methodology for testing network independence to any multivariate nodal attributes can be applied to any exchangeable graph. We prove the consistency of test statistics and demonstrate better performance than any other model-based, global test statistics. Simulation in a variety of networks shows higher power of test and also application to dMRI networks will be followed. 
	\end{abstract}
	
	\noindent%
	{\it Keywords:} distance correlation, multiscale generalized correlation, diffusion maps, exchangeable graph, stochastic block model 
	\vfill
	
	\newpage
	\spacingset{1.45} % DON'T change the spacing!
	\section{Introduction}
	\label{sec:intro}
	
	Network, a collection of nodes and edges between them, has been a celebrated area of study over a field of psychology, information theory, biology, statistics, economics, etc \textcolor{red}{references}. The relationship between a way that a pair of nodes is connected and the values of their attribute is a common interest in network analysis. According to an assumption on how they are related each other, there has been a lot of efforts to represent a network as a function of nodal attributes \textcolor{red}{references} or model an outcome of nodal attribute variables through their underlying network structures \textcolor{red}{references}. However, it is very obscure to determine which one should be put as a dependent variable or even whether networks are truly related to nodal attributes and how they are, if any. A fundamental difficulty is that network often does not have a natural structure and networks of individuals in the network, nodes, are dependent on others. Thus it is not easy to intuitively think of how to represent network as a node-wise random variable. \cite{fosdick2015testing} therefore suggested estimating network factors which are believed to represent each node's locations in network space and using them to test independence network topology and nodal attributes by implementing standard statistical testing between the factors and attributes. Through controlling the dimension of latent factors they make up the constraints due to parametric modeling. However statistical modeling on networks still rely on the assumption that all the nodes in network would follow the same pattern of dependence and also same amount of dependence. This might not be true for always. We develop a nonparametric test statistic which is also sensitive to nonlinear and local dependence pattern.  
	
	Throughout this paper, assume that we are given an unweighted and undirected, connected network $\boldsymbol{G}$ comprised of $n$ nodes, for a fixed $n \in \mathbb{N}$. Even though we assume that $\boldsymbol{G}$ is an undirected and unweighted network, we are able to extend all of the theory here to directed and even weighted network. An adjacency matrix of a given network, denoted by $\boldsymbol{A} = \{A_{ij} : i,j= 1,..,n \}$, is often introduced to formalize this relational data of network, where $A_{ij} = 1$ if node $i$ and node $j$ are adjacent each other and zero otherwise. Let us introduce a $m$-variate ($m \in \mathbb{N}$) variable for nodal attributes $\boldsymbol{X}  \in \mathbb{R}^{m}$ which we are interested in. Investigating correlation between $\boldsymbol{G}$ and $\boldsymbol{X}$ and testing whether their distributions are independent or not is the key focus in our study. An observed network $\mathbf{G}$ can represent social network within a school and $\boldsymbol{X}$ is students' grades or heights, for example; or $\mathbf{G}$ is a neuronal network in human brain and $\boldsymbol{X}$ is a few factors of personality. 
	
	In this paper without modeling nor estimating networks, we develop multiscale test statistics which are robust to both nonlinearity and high dimensionality. Multiscale is not avoidable because we view relationship or distance over network as dynamic process so we choose the optimal scale where distance in network space and distance in terms of attributes become most correlated each other. Thus basically we define a coordinate over network at each time in process and test independence to attributes $\mathbf{X}$. In the next methodology section \ref{sec:method}, we are going to define such multiscale distance and its properties, and test statistic using multiscale distance metrics will be followed. In section \ref{sec:sim}, we demonstrate best performance of our method compared to the existing under various circumstances through numerical results. Real data example in section \ref{sec:real} show one of the applications among many.  
	
	%%%%%%%%%%%%%%%%%%%%%%%%%%%%%%%%%%%%%%%%%%
	
	\bigskip
	\section{Methodology}
	\label{sec:method}
	
First step in testing network independence to attributes is figuring out a variable which configurates location of nodes over network space. We hope that our observed network be a representative realization of this variable so that our test results can be generalized to population network. To be specific a pair of our observations for each variable we are going to test should be independent and identically-distributed(\textit{i.i.d}) realization to guarantee representativeness and avoid redundant intra-dependence.  
	
	\subsection{Exchangeable Graph}
	
	Assuming that we are given one network, equivalently one \textit{graph} comprised of nodes and edges. If you consider each edge as \textit{i.i.d}, then observed adjacency matrix is a random sample of a single parameter which is the probability of having edge or not. However, in this case, the resulting network model only depends on number of edges (\cite{orbanz2015bayesian}) and does not cover undirected or no self-loop networks. Instead of assuming \textit{i.i.d} of edges right away, we are going to assume underlying distribution of edges and then consider a set of edges as \textit{i.i.d} conditional on random function. This idea comes from exchangeable representation of edges.
	The property of exchangeability is closely related to the use of \textit{i.i.d} random variable. Network or graph $\mathbf{G}$ is called exchangeable if and only if its adjacency matrix $\mathbf{A}$ is jointly exchangeable (\cite{orbanz2015bayesian}). 
	
	\begin{definition}[2-array exchangeability]
		\label{exchangeability}
		A random 2-array $(A_{ij})$ is called $\mathbf{\mbox{jointly exchangeable}}$ if 
		$$(A_{ij}) \stackrel{d}{=} (A_{\sigma(i) \sigma(j)})$$
		for every permutation $\sigma$ of $n$,
		and separately exchangeable if 
		$$(A_{ij}) \stackrel{d}{=} (A_{\sigma(i) \sigma^{\prime}(j) })$$
		for every pair of permutation $\sigma, \sigma^{\prime}$ of $n$.
	\end{definition}
	
	However, exchangeability itself cannot guarantee being \textit{i.i.d}. Fortunately, thanks to the celebrated  \textit{de Finetti}(\ref{finetti})'s representation theorem, it has been proven that there exists a random probability measure $\eta$ on random variable $\mathbf{Z}$ that a sequence of $Z_{1}, Z_{2}, \ldots $ are \textit{i.i.d} conditional $\eta$ \textit{if and only if} the sequence is exchangeable (\cite{orbanz2015bayesian}, \cite{caron2014sparse}). \textit{Aldous-Hoover theorem}(\ref{Aldous_Hoover}) is the representation theorem of 2-array exchangeable array, which is useful to explain jointly exchangeable adjacent matrix. Exchangeable graph is commonly called \textit{graphon}. Exchangeable graphon is defined through a random measurable functions (\cite{chan2013estimation}).
	
\begin{definition}[graphon]
		\label{graphon}
		
		A \textit{graphon} with $n (\in \mathbb{N})$ nodes is defined as a function of a symmetric measurable function $g : [0,1]^2 \rightarrow [0,1]$ with input of $u_{i} \overset{i.i.d}{\sim} Uniform[0,1], i = 1,2,... ,n$. 
		Let $\mathbf{A}$ be an adjacency matrix of graphon. Then for any $i < j, i,j=1,2,...,n$:	
\begin{equation}
	Pr \big(   A_{ij} = 1 \big| u_{i}, u_{j} \big) = g \big(  u_{i}, u_{j} \big)
\end{equation}
\end{definition}
By \textit{Aldous-Hoover theorem}, we can now better represent exchangeable network through measurable function, but we are still halfway done in case of undirected network where $A_{ij} = A_{ji}$  $(i,j=1,2,... , n)$.  Under undirected network where self-loop is still allowed, we can represent $\{ A_{ij} : i \leq j \}$ as a function of $g$ of $\{ u_{i}\}$. 
	
\begin{equation}
( A_{ij} )  =  (   A_{\sigma(i) \sigma(j)}  ) \Longleftrightarrow A_{ij} \overset{ind}{\sim} Bern\big(  g(u_{i}, u_{j}) \big), i < j
\end{equation}  
	
Networks from widely used graphical model are exchangeable. One of the most popular models is Stochastic Block Model (SBM). 
The SBM, in the simplest setting, assumes that each of $n$ nodes in graph $\boldsymbol{G}$ belongs to one of $K \in \mathbb{N} (\leq n)$ blocks or groups. Block affiliation is important in that the probability of having edges between a pair of nodes depends on which blocks they are in.  Assume that latent variables corresponding to block affiliation follow $Z_{1}, Z_{2}, ... , Z_{n} \overset{i.i.d.}{\sim} Multinomial\big( \pi_{1}, \pi_{2}, ... , \pi_{K} \big)$. Then the upper triangular entries of $\mathbf{A}$ are independent and identically distributed conditional on $\{\mathbf{Z}\}$:
	\begin{equation} 
	A_{ij} \overset{i.i.d.}{\sim} Bern\big( \sum\limits_{k,l=1}^{K} p_{kl} I\big( Z_{i} = k, Z_{j} = l  \big)    \big), \forall  i < j.
	\end{equation}

The above distribution can also be represented through some random function $g : [0,1]^1 \rightarrow [0,1]$. Let $W_{1}, W_{2}, ... , W_{n} \overset{i.i.d.}{\sim} Unif[0,1]$ and $g\big( W_{i}, W_{j} \big) = \sum\limits_{k,l=1}^{K} p_{kl} I \big( W_{i} \in [\sum\limits_{j=0}^{k-1} \pi_{j}, \sum\limits_{j=0}^{k} \pi_{j}   ] , W_{j} \in [\sum\limits_{j=1}^{l-1} \pi_{j}, \sum\limits_{j=1}^{l} \pi_{j}  ]  \big)$, where $\pi_{0} = 0$ and $\sum\limits_{j=0}^{K} = 1$ 
\begin{equation} 
A_{ij} \overset{i.i.d.}{\sim} Bern \big( g(W_{i}, W_{j})  \big), \forall i < j
\end{equation}

	
\subsubsection{\textcolor{red}{graphon}}
	
Graphon has been studied widely as a limit of random graphs \textcolor{red}{(ref : )}. However, despite its advantage on simple representation, is either empty or dense. Thus it fails to represent real network data where sparsity or scale-free distribution if fairly common. In addition to graphon,we introduce a concept of \textit{graphex}, first proposed by \cite{veitch2015class}, which is more generalized version of graphon and also includes sparse exchangeable graphs \cite{caron2014sparse}. A trick used in \cite{caron2014sparse} is representing a network as a point process on $\mathbb{R}^2_{+}$ based on \textit{Kallengerg Representation Theorem}\cite{kallenberg1990exchangeable}. Like we are able to represent $\{ A_{ij} \}$ through random transformation of \textit{i.i.d} uniform variables, jointly exchangeable point processing network also can be represented via a random function of \textit{i.i.d} unit rate Poisson process and uniform variables. 
	To be specific, undirected graph on a point process on $\mathbb{R}^2_{+}$ can be thought of 
	\begin{equation}
	\mathbf{A} = \sum\limits_{i,j} A_{ij} \delta_{( \theta_{i}, \theta_{j})} 
	\end{equation}	
	,where $A_{ij} = A_{ji} \in \{ 0 , 1  \}$ with $\theta_{i} \in \mathbb{R}$, $i,j = 1,2,...$. We can think \texttt{Node} $i$ is embedded on real line, at $\theta_{i} \in \mathbb{R}_{+}$. 	
	
	
	\begin{definition}[Joint exchangeability on point process]
		\label{point}
		Let $h > 0$ and  $V_{i} = [h(i-1), hi ]$ for $i \in \mathbb{N}$ then
		\begin{equation}
		\big( A( V_{i} \times V_{j}  )   \big)  \stackrel{d}{=} \big( A( V_{\sigma(i)} \times V_{\sigma(j)}     \big)
		\end{equation}	
		for any permutation $\sigma$ of $\mathbb{N}$.		
	\end{definition}
	
	
	
	\begin{definition}[graphex \cite{kallenberg1990exchangeable}]
		\label{graphex}
		
		
	\end{definition}
	

	
%%%%%%%%%%%%%%%%%%%%%%%%%%%%%%%%%%%
\subsection{Multiscale Distance Metrics}	


\subsubsection{Diffusion maps and diffusion distance}
	
There have been a lot of efforts to represent the network in terms of a summarizing network factor (\cite{hoff2002latent}) or some meaningful coefficients, e.g. centrality (\cite{mantzaris2013dynamic}, \cite{sporns2007identification}). However, there has been no node-wise variable which provides a configuration of node over network space without losing any information. \cite{coifman2006diffusion} proposed a meaningful multiscale geometries of data called \textit{diffusion maps} while keeping information on every local relation. Diffusion map is constructed via Markov chain on graph. Without any model assumption on graph, an adjacency matrix $\boldsymbol{A}$ acts as a kernel, representing a similarity between each node in $\boldsymbol{G}$; thus we do not have to estimate anything in order to obtain multidimensional representation of network topology. 
	
Let $(\boldsymbol{G}, \mathcal{A}, \mu)$ be a measure space. Throughout all of the arguments, assume that we have a countable node set with size of $n \in \mathbb{N}$. The vertex set of network $\boldsymbol{G}$ is the data set of nodes and edges and $\mathcal{A}$ is a set of a pair of nodes $\{(i,j) : v_{i}, v_{j} \in V(\boldsymbol{G}) \}$. A measure of $\mu$ which represents a distribution of the nodes on $\boldsymbol{G}$, is equivalent to an adjacency matrix $\boldsymbol{A}$. A transition matrix $\mathbf{P} = \{P[i,j] : i,j=1,...,n \}$ in Markov chain on $\boldsymbol{G}$, which represents the probability that flow or signal goes from Node $i$ to Node $j$, is defined as below:
	
\begin{equation}
P[i,j] = A_{ij} \big/ \sum\limits_{j=1} A_{ij}
\end{equation}
	
A transition matrix $\mathbf{P}$ is a new kernel of a Markov chain of which element $P[i,j]$ represents the probability of travel from Node $i$ to Node $j$ in one time step. A corresponding probability in $t$ step is given by the $t$ th ($t \in \mathbb{N}$) power of $P$. Now we assume that diffusion process occurs within a given graph with this transition probability at each time step. Distance between a pair of nodes at each time is called \textit{diffusion distance}. That is we have distance between every pair of nodes throughout diffusion process. How to derive diffusion distance over a directed network or weighted network is provided in \cite{tang2010graph}. Other than a transition matrix, we need a stationary probability $\boldsymbol{\pi} = \{\pi(1), \pi(2), ... , \pi(n) \}$ of which $\pi(i)$ represents the probability that the diffusion process in the end is stuck in Node $i$ regardless of the starting state. In our setting, $\pi(i)$ is assumed to be proportional to the degree of Node $i$, i.e. $\pi(i) = \sum\limits_{j=1}^{n} A_{ij} \big/ \sum\limits_{i=1}^{n}\sum\limits_{j=1}^{n} A_{ij}$ ($i=1,2,..., n$).   
	
For each time point $t \in \mathbb{N}$, we can define a diffusion distance $C_{t}$ defined on each pair of nodes given by :	
\begin{equation}
\label{eq:diffusion}
\begin{split}
C^2_{t}[i,j] & = \sum\limits_{w =1}^{n} \big( P^{t}[i,w] - P^{t}[j,w]  \big)^{2} \frac{1}{\pi(w)} = \sum\limits_{w=1}^{n} \left(  \frac{P^{t}[i,w]}{\sqrt{\pi(w)}} - \frac{P^{t}[j,w]}{\sqrt{\pi(w)}}   \right)^2 \\ & = \parallel P^{t}[i, \cdot] - P^{t}[j, \cdot]  \parallel^2_{L^{2}(\boldsymbol{G}, d\mu / \pi)  }
\end{split}
\end{equation}
As diffusion time $t$ increases, distance matrix $C_{t}$ is more likely to take into account distance between two nodes which are difficult to reach each other. Basically diffusion distance at fixed time $t$ measures the chance that we are likely to stay between Node i and Node j at $t$ step on our journey of all other possible paths. The higher chance is, the smaller distance between two is. Since unlikely to adjacent relation or geodesic distance, it takes into account every possible path between two nodes, diffusion distance well reflects the connectivity. Simply speaking, connectivity between two nodes is higher if we need to eliminate more number of nodes to disconnect these two. It is more robust metric to the unexpected edges than geodesic distance. Often a set of nodes with higher connectivity have a higher propensity of having edges within this set and they are likely to form a cluster. This kind of cluster can be considered as block in SBM. 
	
Diffusion distance of $\boldsymbol{G}$ defined as above can be represented via a spectral decomposition of its transition matrix $P$. That is, we can derive diffusion distance using its eigenvectors and eigenvalues. The spectral analysis on diffusion distance or diffusion maps have been studied mainly for its usefulness for nonlinear dimensionality reduction (\cite{coifman2006diffusion}, \cite{lafon2006diffusion}). 
Recall that diffusion distance at time $t$, $C_{t}$ is a functional $L^2$ distance, weighted by 1/$\pi$ in \ref{eq:diffusion}. If we transform the way to represent $C_{t}[i,j]$ slightly, we are able to obtain an orthonormal basis of $L^{2}(\mathbf{G}, d\mu / \pi)$ via eigenvalues and eigenvectors. 
Since an adjacency matrix $A$ does not guarantee a symmetric of $P$, define a symmetric kernel $\boldsymbol{Q} = \boldsymbol{\Pi^{1/2} P \Pi^{-1/2}},$ where $\mathbf{\Pi}$ is a $n \times n$ diagonal matrix of which $i$th diagonal element is $\pi(i)$. Under compactness of $P$, $\boldsymbol{Q}$ has a discrete set of real nonzero eigenvalues $\{ \lambda_{r} \}_{r = \{1,2,...,q \}}$ and a set of their corresponding orthonormal eigenvectors $\{ \psi_{r} \}_{r = \{1,2,..., q \} },$ i.e. $Q[i,j] = \sum\limits_{r=1}^{q} \lambda_{r} \psi_{r}(i) \psi_{r}(j)$ ($1 \leq q \leq n$).  
Since $P[i,j] = \sqrt{\pi(j) / \pi(i) } Q[i,j]$, $P[i,j]= \sum\limits_{r=1}^{q} \lambda_{r} \{ \psi_{r}(i) / \sqrt{\pi(i)}  \} \{ \psi_{r}(j) \sqrt{\pi(j)} \} := \sum\limits_{r=1}^{q} \lambda_{r} \phi_{r}(i) \{ \psi_{r}(j) \sqrt{\pi(j)} \}$, where $\phi_{r}(i) := \psi_{r}(i) / \sqrt{\pi(i)}$. Then from $\sum\limits_{r=1}^{q} \psi_{r}(j) \sqrt{\pi(j)} = 1$ for all $j \in \{1,2,...,n\}$, 
we can represent the diffusion distance as: 
	
\begin{equation}
\begin{split}
	C^2_{t}[i,j] & = \sum\limits_{r=1}^{n} \lambda^{2t}_{r} \big( \phi_{r} (i) - \phi_{r}(j)   \big)^2   \\ &  = \parallel P^{t}[i, \cdot] - P^{t}[j, \cdot]  \parallel^2_{L^{2}(\boldsymbol{G}, d\mu / \pi)  }
	\end{split}
\end{equation}
	
	That is,
	
	\begin{equation}
	C_{t}[i,j] = \parallel \boldsymbol{U}_{t}(i) - \boldsymbol{U}_{t}(j) \parallel
	\end{equation}
	, where 
	
	\begin{equation} 
	\boldsymbol{U}_{t}(i) = \begin{pmatrix} \lambda^{t}_{1} \phi_{1}(i) \\ \lambda^{t}_{2} \phi_{2} (i)  \\ \vdots \\ \lambda^{t}_{q} \phi_{q}(i) \end{pmatrix} \in \mathbb{R}^{q}.
	\end{equation}
	
	
	Now we call the family of $q$-variate($q \leq n$) diffusion maps $\{ U_{t} \}_{t \in \mathbb{N}} $, of which Euclidean distance is diffusion distance.
	
	
\subsubsection{Properties of diffusion maps under exchangeable graphs}

We hope that diffusion maps are multivariate configuration of each nodes of which distance between two nodes represents distance on network space. However, due to the inter-correlated construction of $U$, e.g. $i$th subject's diffusion depends on others in the given network, it is hard to say that the observed diffusion coordinates of $n$ subjects are independent samples right away. To talk about independence of $U$, we need more general concept of exchangeable graph explained in the previous section. 
	
\begin{lemma}[Exchangeability and i.i.d. of $A$ in graphon]
	\label{lemma1}
Assume that a connected, undirected and unweighted graph $\mathbf{G}$ is a graphon. Then 2-array of $\{ A_{ij} : i = 1,2,... ,n , i \leq j \}$ are i.i.d. conditioning on random link function $g : [0,1]^2 \rightarrow [0,1]$. Thus for fixed row (column) of $\mathbf{A}$, $\{ A_{i1}, A_{i2}, ... , A_{in} \}$, $i \in \{ 1,2,... , n \}$ are conditionally i.i.d. on random link function $g$.  
\end{lemma}
	
	
	
\begin{lemma}[Exchangeability and i.i.d. of $A$ in graphex]
		\label{lemma2}
		Assume that a connected, undirected and unweighted graph $\mathbf{G}$ is a graphex. Then 2-array of $\{ A_{ij} : i = 1,2,... ,n , i \leq j \}$ \textcolor{red}{need cautions}
\end{lemma}
	
	
	From the above Lemma \ref{lemma1}, \ref{lemma2}, we can also prove exchangeability and conditional i.i.d. of diffusion maps at each time point. 
	
	\begin{lemma}[Exchangeability and iid of $U$]
		\label{lemma3}
		Assume that a connected, undirected and unweighted graph $\mathbf{G}$ is a graphon, i.e. any exchangeable random graph from an infinite graph. Then its transition probability $P_{ij}$ so thus  diffusion maps at fixed time $t$ also exchangeable conditional on link function of graph. Furthermore, by [de Finetti]'s Theorem\ref{finetti}, we can say that such diffusion maps at $t$ are conditionally i.i.d given random probability measure $\eta$ on $U_{t}$ and random link function $g$.    
	\end{lemma}
	
	
	Lemma \ref{lemma3} above provides us i.i.d. one-parameter family of $\{ \mathbf{U}_{t} \}_{t \in \mathbb{N}}$ conditional on a random probability measure of $\mathbf{U}_{t}$ and a random link function of $g$.
	
	
%%%%%%%%%%%%%%%%%%%%%%%%%%%%%%%%%%%%%%%%%%%%%%%%%%%%%%%%%%%%%%%%
\newpage
\subsection{Applying to Multiscale Generalized Correlation}
	
\subsubsection{Distance correlation and its multiscale version}
	
Relationship between network and nodal attributes often exhibits local or nonlinear properties. Moreover, dimension of spectrum of network$(q)$ increases as a sample size increases. Unfortunately, widely used correlation measures often fail to capture nonlinear associations or multivariate associations. \cite{szekely2007measuring} extended pairwise constructed generalized correlation coefficient and developed a novel statistics called distance correlation (\texttt{dCov}) as a measure for all types of dependence between two random vectors in any dimension. Let us first start from a general setting that we are given $n \in \mathbb{N}$ pairs of random samples $\{ (x_{i}, y_{i}) : x_{i} \in \mathbb{R}^{q}, y_{i} \in \mathbb{R}^{m}, i = 1,...,n \}$. Define $C_{ij} = \parallel x_{i} - x_{j} \parallel$ and $D_{ij} = \parallel y_{i} - y_{j} \parallel$ for $i,j=1,...,n$, where $\parallel \cdot \parallel$ denotes Euclidean distance defined on any vectors.   
Distance correlation (\texttt{dCor}) is defined via distance covariance (\texttt{dCov}) $\mathcal{V}^2_{n}$ of $\boldsymbol{X}$ and $\boldsymbol{Y}$, which is the following: 
	
\begin{equation}	 
\mathcal{V}^2_{n}(\boldsymbol{X}, \boldsymbol{Y}) = \frac{1}{n^2} \sum\limits_{i,j=1}^{n} \tilde{C}_{ij} \tilde{D}_{ij}
\end{equation}
, where $\tilde{C}$ and $\tilde{D}$ is a doubly-centered $C$ and $D$ respectively, by its column mean and row mean. Distance correlation $\mathcal{R}^{2}_{n}(\boldsymbol{X}, \boldsymbol{Y})$ is a standardized \texttt{dCov} by $\mathcal{V}^2_{n}(\boldsymbol{X}, \boldsymbol{X})$ and $\mathcal{V}^2_{n}(\boldsymbol{Y}, \boldsymbol{Y}).$
	
\begin{equation}	 
\mathcal{R}_{n}^{2} (\boldsymbol{X}, \boldsymbol{Y}) = \frac{\mathcal{V}^2_{n} (\boldsymbol{X}, \boldsymbol{Y}) }{\sqrt{\mathcal{V}^2_{n} (\boldsymbol{X}, \boldsymbol{X}) \mathcal{V}^2_{n} (\boldsymbol{Y}, \boldsymbol{Y}) } }
\end{equation}
	
On the other hand, a modified distance covariance (\texttt{MCov}) $\mathcal{V}^*_{n}$ and a modified distance correlation (\texttt{MCorr}) $\mathcal{R}^{*}_{n}$ for testing high dimensional random vectors were also proposed in \cite{szekely2013distance}.   
However, \texttt{dCov} and even \texttt{MCorr} still perform not very well in existence of various nonlinear dependency and under existence of outliers (Cencheng). Out of this concern, Cencheng at al (2016) developed Multiscale Generalized Correlation (\texttt{MGC}) by adding local scale on correlation coefficients. Multiscale version of distance covariance $\{ { {\mathcal{V}^{*}}^2_{n} }   \}_{kl}$ is defined as following : 
	
\begin{equation}
\label{eq:MGC}
{\mathcal{V}^{*}}^2_{n} (\boldsymbol{X}, \boldsymbol{Y})_{kl} = \frac{1}{n^2} \sum\limits_{i,j=1}^{n} \tilde{C}_{ij} \tilde{D}_{ij} I \big( r(C_{ij}) \leq k \big) I \big( r(D_{ij})) \leq l  \big), \quad k,l=1,2,..., n 
\end{equation}
,where $r(C_{ij})$ ($r(D_{ij})$) is a rank $\mathbf{x}_{i}$ ($\mathbf{y}_{i}$) relative to $\mathbf{x}_{j}$ ($\mathbf{y}_{j}$). We basically truncate each pairwise element of distance covariance with respect to rank in terms of (Euclidean) distance. Note that if $k=l=n$, $\mathcal{V}^2_{n}$ and ${\mathcal{V}^{*}}^2_{n}$ are equivalent. Since we call all family of $\{  {\mathcal{R}^{*}}^2_{n} \}_{k,l = 1,2,...,n}$ as \texttt{MGC} and choose the optimal set of neighborhood choice of $(k,l)$, \texttt{MGC} is more generalized version of \texttt{dCor}. Its superiority and consistency have been proven against all alternatives Cencheng. Particularly, in simulation\ref{sec:sim} we are going to show in which pattern of underlying dependency exists \texttt{MGC} is much more sensitive than global scale of statistics. 
	
	
\subsubsection{Choice of proper metric on network}

Returning to the problem of network setting, the fundamental problem is in measuring all types of dependence between $\boldsymbol{G}$ and $\boldsymbol{X}$ via \texttt{MGC}, we are required a \textit{i.i.d.} vertex-wise coordinates of which Euclidean distance measures a distance between them. You might first propose directly using a column of an adjacency matrix so that we have a $n$-pair of observations $\big\{ \big( \boldsymbol{A}_{i \cdot} , \boldsymbol{X}_{i} \big) : \boldsymbol{A}_{i \cdot} = (A_{i 1} , ... , A_{i n} ), \boldsymbol{X}_{i} \in \mathbb{R}^{m}, i=1,...,n  \big\}.$ In an undirected graph, $\{ \mathbf{A_{i \cdot}}  \}$ cannot be independent. Even if it is in e.g. directed graph, Euclidean distance between $\{ \boldsymbol{A}_{i \cdot} : i =1, ... , n \}$ is not a proper metric over network space. Let us introduce a simple example. Let a given network $\boldsymbol{G}$ having 8 nodes be an unweighted, directed network and possibly having self-loop. Let $\boldsymbol{A}$ be its $8 \times 8$ binary adjacency matrix. Assume \texttt{Node 1}, \texttt{Node 4} and \texttt{Node 8} have the following row entries:
	
\begin{equation}
	\begin{gathered}
	\boldsymbol{A}_{1 \cdot} = \left( \begin{array}{rrrrrrrr} 1 & 1 & 1 & 1 & 1 & 1 & 1 & 1 \end{array} \right) \\
	\boldsymbol{A}_{4 \cdot} = \left( \begin{array}{rrrrrrrr} 1 & 1 & 1 & 1 & 0 & 0 & 0 & 0 \end{array} \right) \\
	\boldsymbol{A}_{8 \cdot} = \left( \begin{array}{rrrrrrrr} 1 & 0 & 0 & 0 & 0 & 0 & 0 & 0 \end{array} \right)
	\end{gathered}
\end{equation}
,which results $\parallel \boldsymbol{A}_{1 \cdot} -\boldsymbol{A}_{4 \cdot} \parallel^2 = 4$, $\parallel \boldsymbol{A}_{1 \cdot} -\boldsymbol{A}_{8 \cdot} \parallel^2 = 7$,and $\parallel \boldsymbol{A}_{4 \cdot} -\boldsymbol{A}_{8 \cdot} \parallel^2 = 3.$ Accordingly, $\parallel \boldsymbol{A}_{4 \cdot} -\boldsymbol{A}_{8 \cdot} \parallel  < \parallel \boldsymbol{A}_{1 \cdot} -\boldsymbol{A}_{4 \cdot} \parallel$. However, you can easily see that this does not make sense because \texttt{Node 4} and \texttt{Node 8} are connected each other only through \texttt{Node 1}. 
Therefore instead of using an adjacency matrix directly, we are considering embedding a vertex $v \in V(\boldsymbol{G})$ into its diffusion map of $\boldsymbol{U}$ and apply Euclidean distance metric, which is exactly a diffusion distance. As explained before, its Euclidean distance takes into account all possible paths between every pair of node and measure the connectivity between them. Unlike in the other metrics in network, i.e. adjacency matrix or geodesic distance, triangle inequality holds in diffusion distance (proof in [Appendix]). 
\begin{corollary}[Triangle inequality]
	\label{corollary1}
		For fixed time $t$, let $C_{t} : V(\mathbf{G})^2 \rightarrow \mathbb{R}_{+}$ be a diffusion distance defined on a pair of vertices in any connected and undirected graph $\mathbf{G}$. Then for any $v, w, z \in V(\mathbf{G})$,  
		\begin{equation}
		C_{t}(v,z) \leq C_{t}(v,w) + C_{t}(w,z)
\end{equation}
\end{corollary}	
	
Thanks to these properties of diffusion maps, we have better interpretation of its Euclidean distance so that we will use it to the distance matrix in \texttt{MGC}.
	
\subsubsection{One parameter family of test statistic}

We have discussed one-parameter family \textit{i.i.d.} representation of network structures, called diffusion maps, and also discussed when given \textit{i.i.d}. pair of observations of two variables, how distance correlation and its multiscale version test independence between these two. Now it is time to combine these two to test independence between network structures and nodal attributes efficiently and effectively.
	
\begin{theorem}[MGC of testing independence]
 \label{theorem1}
	Assume that a connected, undirected and unweighted graph $\mathbf{G}$ is an exchangeable graph.  Assume that we are given $n$-pair of observations $\{ (u^{(t)}_{i}, x_{i}): i = 1,2,... , n  , t \in \mathbb{N} \}$. Then $u_{t}(i) \overset{i.i.d.}{\sim} f_{U_t} \big(  \eta, g \big)$, $t \in \mathbb{N}$ and $x_{i} \overset{i.i.d.}{\sim} f_{X}$, where $f_{U_t} \big( \eta, g \big)$ are conditional distribution function given a link function $g$ and a random probability measure $\eta_{t}$ of $U_t$. Then \texttt{MGC} applied to these pair of data is theoretically consistent against all dependent alternatives in testing :
		$$H_{0} : f_{U_t \cdot X} = f_{U_t} \cdot f_{X}$$
\end{theorem}
	
Since $\boldsymbol{U}$ provides a configuration of vertices in $\boldsymbol{G},$ the above hypothesis implies testing independence between the configuration of vertices in network space and in attribute space, but as a function of a link function $g$ and a random function (variable) of $\eta$ of $\mathbf{U}$.
	
\begin{remark}
	Roughly speaking, we can say that diffusion maps are i.i.d. function of a link function $g$ and a random function of $\mathbf{U}$, $\eta$. Thus testing independence between conditional $U$ and $X$ can be considered as testing independence between $f \big( g, \eta \big)$ and $X$. A link function $g$ concerns the distribution of edges and a random function $\eta$ concerns nature distribution of diffusion maps. Our testing basically examines whether how edges are constructed and how diffusions(propagation) process are correlated to nodal attributes.  
	\end{remark}
	
%%%%%%%%%%%%%%%%%%%%%%%%%%%%%%%%%%%%%%%%%%%%%%%%%%%%%%%%%%%%%%%%
\section{Simulation Study}
\label{sec:sim}
	
In simulation studies presented in this paper, we make a comparison of estimated testing power across various multivariate independence test statistics: \texttt{MGC}, \texttt{dCov}, Heller-Heller-Gorfine (\texttt{HHG}) (\cite{heller2012consistent}), and likelihood ratio test of Fosdick and Hoff (\texttt{FH}). For computing statistical power, we used type I error $\alpha = 0.05$ and obtain p-values of each sample network via permutations. For fair comparison between these testing methods, we also present a additive model of latent factors, which is mostly targeted by \texttt{FH}.   
	
\subsection{Stochastic Block Model}

We mentioned in the Introduction that latent network model is very common followed by the assumption of local independence. Stochastic Block Model (SBM) is one of the most popular and also useful network generative model, especially as a tool for community detection \cite{karrer2011stochastic}. 
	
\subsubsection{[sim1.1] simplest two block model}
	
	\begin{equation}
	\begin{gathered}
	X_{i} \overset{i.i.d}{\sim} Bern(0.5), i = 1,... , n \\ 
	Z_{i}  \sim  \left\{  \begin{array}{cc} Bern(0.6) & X_{i} = 0 \\ Bern(0.4) & X_{i} = 1  \end{array} \right. \\
	A_{z_{i}, z_{j}} \sim Bern \left[  \begin{array}{cc}   0.4 & 0.1  \\ 0.1 & 0.4 \end{array}  \right]
	\end{gathered}
	\end{equation}
	
	
	\textcolor{red}{figure which compares power of MGC/dCov/hhg/FH}
	
	
	
\subsubsection{[sim1.2] simplest three block model}
	
	
	\begin{equation}
	\begin{gathered}
	X_{i} \overset{i.i.d}{\sim} Multi(1/3, 1/3, 1/3), i = 1,2, ... , n \\ 
	Z_{i}  \sim  \left\{  \begin{array}{ccc} Multi(1/2, 1/4, 1/4) & X_{i} = 1 \\ Multi(1/4, 1/2, 1/4) & X_{i} = 2 \\ Multi(1/4, 1/4, 1/2) & X_{i} = 3  \end{array} \right. \\
	A_{z_{i}, z_{j}} \sim Bern \left[  \begin{array}{ccc}   0.5 & 0.2 &  0.3  \\ 0.2 & 0.5 & 0. 2  \\ 0.3 & 0.2 & 0.5  \end{array}  \right]
	\end{gathered}
	\end{equation}
	
	\textcolor{red}{comparison to patter of power of MGC/dCov/hhg/FH from "simple" three block graph - figures in appendix}
	
	
	\textcolor{red}{explain briefly why local optimal performs better in this case}
	
	
	\textcolor{red}{figure which compares power of MGC/dCov/hhg/FH}
	
	
	
\subsection{[sim] other version of graph generation}
	
	
\subsubsection{[sim2.1] degree-corrected stochastic block model}
	
Under SBM, we assume that all nodes within the same block have the same expected degree. However, this block model is limited by homogeneous distribution within block and provides a poor fit to networks with hubs or highly varying node degrees within blocks or communities, which are common in practice. On the other hand, the Degree-Corrected Stochastic Block model (DCSBM) adds an additional set of parameter, often denoted by $\theta$, to control the node degrees. This model allows variation in node degrees within a block while preserving the overall block community structure. 
	
	
	
	
	
\textcolor{red}{figure which compares degree distribution of SBM vs. DCSBM}
	
\subsubsection{\textcolor{red}{[sim2.2] exchangeable graph on Poisson process}}
	
	
	\textcolor{red}{simulation study explained in Veitch and Roy, 2015}
	Sample a (latent) unit rate Poisson process $\Pi$ on $\theta \times \vartheta \in [0, \nu] \times [0, c]$.
	
	\begin{equation}
	\begin{gathered}
	N_{\nu} \sim Poi( c \nu) \\ 
	\{ \theta_{i} \} \big| N_{\nu} \overset{iid}{\sim} Uniform[0, \nu]  \\ 
	\{ \vartheta_{i} \} \big| N_{\nu} \overset{iid}{\sim} Uniform[0,1] \\ 
	\{ X_{i}  \} | N_{\nu} \overset{ind}{\sim} Normal \big( \vartheta_{i}, 0.1 \big)  \\ 
	(\theta_{i}, \theta_{j}) \big| W, \vartheta_{i}, \vartheta_{j} \overset{ind}{\sim} Bernoulli \big( W\big( \vartheta_{i}, \vartheta_{j} \big) \big) 
	\end{gathered}
	\end{equation}
	
	
	
\subsubsection{Additive and multiplicative graph model}
	
\cite{hoff2002latent} proposed a approach of jointly modelling network and its attributes, where networks represent additional structure via sender-specific(or row-specific) and receiver-specific(or column-specific) latent factors.
As a connection to RDPG, this model can also represented as a product of combined latent factor $\tilde{u}$ and $\tilde{v}$. 
	
\begin{equation}
\begin{gathered}
	u_{i} \overset{i.i.d}{\sim} Uniform[0,1] \\ 
	A_{ij}  \overset{i.i.d}{\sim} Bern \big(  ( 1 - u_{i})^2 \times (1 - u_{j})^2    \big)
\end{gathered}
\end{equation}
	
%%%%%%%%%%%%%%%%%%%%%%%%%%%%%%%%%%%%%%%%%%%%%%%%%%%%%%%%%%%%%%%%
\section{Real Data Examples}
\label{sec:real}
	
\subsection{MRI}
	
look into one largest connected component. 
	
$\{ \mathbf{U}_{t} \in \mathbb{R}^{q} \}_{t \in \mathbb{N}}$ $\mathbf{X} = (x,y,z) \in \mathbb{R}^{3}$
	
	(it turns out that $q=$ )?
test what? test independence between brain network and its 3-dimensional locations; test independence between functional location and physical location. 
	
\begin{figure}[H]
		\centering
		\label{fig:mri}
		\includegraphics[width=2in]{../Figure/brain1_x.pdf}
		\includegraphics[width=2in]{../Figure/brain1_y.pdf}
		\includegraphics[width=2in]{../Figure/brain1_z.pdf}
		\caption{Subnetwork of MRI network. Darker colored nodes indicate higher positioned node in terms of $x$-axis(left), $y$-axis(middle), and $z$-axis(right).}
\end{figure}
	
	
%%%%%%%%%%%%%%%%%%%%%%%%%%%%%%%%%%%%%%%%%%%%%%%%%%%%%%%%%%%%%%%%
\section{Discussions}
\label{sec:discussion}
	
Throughout this study, we demonstrate that multiscale network test statistic to test network independence performs well in diverse settings, being supported by thorough theory on distance correlation and diffusion maps. 
Testing independence is often the very first step in investigating relationship between network topology and nodal attributes in our interest. It is more likely that we want to know more than binary decision of rejecting or not rejecting the hypothesis. Multiscale test statistics due to both neighborhood choice $\{ (k,l)  \}$ and time spent in diffusion processes $\{ t \}$ provides us a hint on latent dependence structure as well.  
	
Due to the ambiguousness of saying \textit{optimal}, our work has some limitations; we do not suggest any theoretically supported tools to select the \textit{optimal} time to obtain p-values of test statistics or maybe we want a family of p-values as a function of $t$. Further research can be focused on restoring true dependence pattern or estimating \textit{optimal} scale from a family of statistics. On the other hand, obtaining a full family of statistics are also computationally infeasible; at every Markov process, we chose the optimal region of neighborhood. As an ad hoc, we chose $t$ with highest power or lowest p-values from 1 to 10 for our simulation \ref{sec:sim}.  

Other than these computational issues, someone might be uncomfortable about being conditioned by a random function of $g$ and $\eta$.  However, this is inevitable for arguing sample properties of being \textit{i.i.d.}, which is also very conceptual and impossible to prove. By conditioning diffusion maps $\{ \mathbf{U}_{t} \}$ by unknown network generative model $g(\cdot, \cdot)$ and also unknown diffusion process model $\eta(\cdot)$, we are finally able to assert that our observations are a fair sample eligible for test. 
	
Despite a few shortcomings listed above, a range of applications of \textit{MNT} statistics and multiscale representation of network is very diverse. Especially multiscale version of test would be very useful when indirect networking not directly from edge is not ignorable or cluster membership significantly affects attributes. 
	
Furthermore even though we specifically constraint the statistic into testing independence between network and nodal attributes, we can now do independence testing of two networks with same size by inputting diffusion distance at each time point from each of network in equation \ref{eq:MGC}. This type of test would be useful when we want to show a pair of networks are topologically or structurally independent. For example, we might wonder if network on \textit{Facebook} and network induced by club activity within class or school are independent or not. 
	
%%%%%%%%%%%%%%%%%%%%%%%%%%%%%%%%%%%%%%%%%%%%%%%%%%%%%%%%
\appendix
\section{appendix}
	
\subsection{supplementary figures}

\begin{equation}
\begin{gathered}
	X_{i} \overset{i.i.d}{\sim} Multi(1/3, 1/3, 1/3), i = 1,2, ... , n \\ 
	Z_{i}  \sim  \left\{  \begin{array}{ccc} Multi(1/2, 1/4, 1/4) & X_{i} = 1 \\ Multi(1/4, 1/2, 1/4) & X_{i} = 2 \\ Multi(1/4, 1/4, 1/2) & X_{i} = 3  \end{array} \right. \\
	A_{z_{i}, z_{j}} \sim Bern \left[  \begin{array}{ccc}   0.5 & 0.2 &  0.2  \\ 0.2 & 0.5 & 0. 2  \\ 0.2 & 0.2 & 0.2  \end{array}  \right]
\end{gathered}
\end{equation}
	
	
	
\subsection{[apd2]}
	\begin{itemize}
		\item {\it  Proof of Lemmas and Theorems  \/}
	\end{itemize}
	
\begin{theorem}[de Finetti's Theorem] 
		\label{finetti}
		
1. Let $X_{1}, X_{2}, ...$ be an infinite sequence of random variables with values in a space $\mathbf{X}$. The sequence $X_{1}, X_{2}, ...$ is exchangeable if and only if there is a random probability measure $\eta$ on $\mathbf{X}$ such that the $X_{i}$ are conditionally i.i.d. given $\eta$. 
		
2. If the sequence is exchangeable, the empirical distributions
		
$$\hat{S}_{n} ( . ) := \frac{1}{n} \sum\limits_{i=1}^{n} \delta_{X_{i}} ( .), n \in \mathbb{N}$$
		converges to $\eta$ as $n \rightarrow \infty$ with probability 1.
\end{theorem}
	
\begin{theorem}[Aldous Hoover Theorem]
		\label{Aldous_Hoover}
		
		Let $\mathbf{A} = \{A_{ij}\}, 1 \leq i,j \leq \infty$ be a jointly exchangeable binary array if and only if there exists a random measurable function $f : [0,1]^{3} \rightarrow \mathbf{A}$ such that 
		
		\begin{equation}
		\big(  A_{ij}  \big) \stackrel{d}{=} \left( f \big( U_{i}, U_{j}, U_{ij} \big)  \right)
		\end{equation}
		where $(U_{i})_{i \in \mathbb{N}}$ and $(U_{ij})_{i,j > i \in mathbb{N}}$ with $U_{ij} = U_{ji}$ are a sequence and matrix, respectively, of i.i.d. Uniform[0,1] random variables. 
\end{theorem}
	
	
	
	\begin{proof}[Proof of Lemma \ref{lemma1}]
		By [Aldous-Hoover] Theorem\ref{Aldous_Hoover}, a random array $(A_{ij})$ is jointly exchangeable if and only if it can be represented as follows : 
		
		There is a random function $g : [0,1]^2 \rightarrow [0,1]$ such that 
		
		$$(A_{ij})  \stackrel{d}{=} Bern( g(W_{i}, W_{j}))$$
		, where $W_{i} \overset{i.i.d.}{\sim} Uniform(0,1)$. Thus if $\mathbf{A}$ is an adjacency matrix of an undirected, exchangeable network, for any $i \leq j,$ $i,j = 1,... , n$:
		
		
		\begin{equation}
		\begin{split}
		P \big(  A_{ij} = a_{ij} \big) & = \int P \big( A_{ij} \big| w_{i}, w_{j} \big) Pr(W_{i} = w_{i}) Pr(W_{j} = w_{j}) dw_{i} dw_{j} \\ & = \int_{0}^{1} \int_{0}^{1} g( w_{i},  w_{j})^{a_{ij}} \big( 1- g( w_{i},  w_{j}) \big)^{1-a_{ij}} dw_{i} dw_{j} 
		\end{split}
		\end{equation}
		
		Then within each row, adjacent elements are independent and also identically distributed ,i.e. for fixed row index to $i \in \{1,2,... , n\}$,
		$$P(A_{i1} = a_{i1}, A_{i2} = a_{i2}, ... , A_{in} = a_{in} ) = \prod\limits_{j=1}^{n} P(A_{ij} = a_{ij})$$
		
	\end{proof}
	
	
	\begin{proof}[Proof of Lemma \ref{lemma2}]
		Based on Kallenberg and Exchangeable Graph (KEG) frameworks, introduced in \href{http://arxiv.org/abs/1512.03099}{[Veitch and Roy]}, a random array $(A_{ij})$ is jointly exchangeable if and only if it can be represented as follows : there is a random function $g : [0,1]^2 \rightarrow [0,1]$ such that 
		
		\begin{equation}
		(A_{ij})  \stackrel{d}{=} (A_{v_{i}, v_{j}} )  \stackrel{d}{=} Bern( g( \vartheta_{i}, \vartheta_{j}))
		\end{equation}
		, where $v_{i} \overset{i.i.d.}{\sim} Poisson(1), \vartheta_{i} \overset{i.i.d.}{\sim} Poisson(1), v_{i} \leq \nu, i = 1,2,... , n$, for some pre-specified $\nu >0$ so that finite size graphs can include vertices only if they participate in at least one edges. 
		
		Thus if $\mathbf{A}$ is an adjacency matrix of an undirected, exchangeable network, for any $i \leq j,$ $i,j = 1,... , n$:
		
		
		\begin{equation}
		\begin{split}
		P \big(  A_{ij} = a_{ij} \big) & = \int P \big( A_{ij} \big| v_{i}, v_{j} \big) Pr(V_{i} = v_{i}) Pr(V_{j} = v_{j}) Pr(\vartheta_{i} = \vartheta_{i}) Pr(\vartheta_{j} = \vartheta_{j})   dv_{i} dv_{j} d\vartheta_{i} d\vartheta_{j}   \\ & = \int_{0}^{\tau} \int_{0}^{\tau} \int_{0}^{\infty} \int_{0}^{\infty}  g( \vartheta_{i},  \vartheta{j})^{a_{ij}} \big( 1- g( \vartheta_{i},  \vartheta_{j}) \big)^{1-a_{ij}}  \\ & \quad \times dPois_{1}(x_{1}) \times dPois_{1}(x_{2}) \times dPois_{1}(x_{3}) \times dPoi_{1}(x_{4})  dx_{1} dx_{2} dx_{3} dx_{4}.
		\end{split}
		\end{equation}
		
		
		
	\end{proof}
	
	
	\begin{proof}[Proof of Lemma \ref{lemma3}]
		
		We have shown that for fixed time $t$, diffusion distance is defined as an Euclidean distance of diffusion maps. Diffusion map is represented as follows :
		
		\begin{equation}
		\boldsymbol{U}_{t}(i) = \begin{pmatrix} \lambda^{t}_{1} \phi_{1}(i) & \lambda^{t}_{2} \phi_{2} (i)  & \cdots & \lambda^{t}_{q} \phi_{q}(i) \end{pmatrix} \in \mathbb{R}^{q}.
		\end{equation}
		
		,where $\Phi = \Pi^{-1/2}\Psi$ and $\mathbf{Q}=\mathbf{\Psi}\mathbf{\Lambda}\mathbf{\Psi}^{T} = \mathbf{\Pi}^{1/2} \mathbf{P} \mathbf{\Pi}^{-1/2}$. 
		Thus $\mathbf{P \Pi^{-1/2} \Psi = \Pi^{-1/2} \Psi \Lambda}$. 
		
		Then for any $r$th row ($r \in \{1,2, ... , q \}$, $(q \leq n)$), we can see that $P \phi_{r} = \lambda_{r} \phi_{r}$, where $\phi_{r} = \begin{pmatrix}  \frac{\psi_{r}(1)}{\sqrt{\pi(1)}} & \frac{\psi_{r}(2)}{\sqrt{\pi(2)}} & \cdots & \frac{\psi_{r}(n)}{\sqrt{\pi(n)}} \end{pmatrix}$.
		Therefore to guarantee exchangeability (or i.i.d.) of $\mathbf{U}_{t}$, it suffices to show exchangeability (or i.i.d.) of $\mathbf{P}$.
		
		Assume joint exchangeability of $\mathbf{G}$, i.e. $(A_{ij}) \stackrel{d}{=} \big( A_{\sigma(i) \sigma(j)} \big)$. 
		Since $A_{ij}$ is binary, $\frac{A_{ij}}{\sum\limits_{ij} A_{ij}} = \frac{A_{ij}}{ 1 + \sum\limits_{l \neq j} A_{il}}$. Moreover, $A_{ij}$ and $(1 + \sum\limits_{l \neq j} A_{il})$ are independent given its link function $g$, and $A_{\sigma(i) \sigma(j)}$ and $(1 + \sum\limits_{l \neq j} A_{\sigma(i) \sigma(l)})$ are independent also given $g$.
		
		Then the following joint exchangeability of transition probability holds:
		
		\begin{equation}
		\big( P_{ij} \big) = \left(  \frac{A_{ij}}{1 - A_{ij} + \sum\limits_{j=1}^{n} A_{ij} } \right)  \stackrel{d}{=} \left( \frac{A_{\sigma(i) \sigma(j)} }{1 - A_{\sigma(i) \sigma(j)} + \sum\limits_{\sigma(j) = 1}^{n} A_{\sigma(i) \sigma(j)} } \right) = \big( P_{\sigma(i) \sigma(j)} \big)
		\end{equation}
		
		
		Thus, transition probability is exchangeable. 
		This results exchangeable eigenfunctions $\{ \Phi(1), \Phi(2), , ... , \Phi(n) \}$ 
		where $\Phi(i) := \begin{pmatrix} \phi_{1}(i) & \phi_{2}(i) & \cdots & \phi_{q}(i) \end{pmatrix}^{T}$. Thus diffusion maps at fixed $t$, $\mathbf{U}_{t} = \begin{pmatrix} \Lambda^{t} \Phi(1)  & \Lambda^{t} \Phi(2) & \cdots & \Lambda^{t} \Phi(n)  \end{pmatrix}$ are exchangeable. 
		
		Furthermore by de Finetti's Theorem(\ref{finetti}), we can say that $\mathbf{U}(t) = \{ U^{(t)}_{1}, U^{(t)}_{2}, ... , U^{(t)}_{n} \}$ are conditionally independent on a random probability measure $\eta$. 
	\end{proof}
	
	
	\begin{proof}[Proof of Theorem \ref{theorem1}]
	\end{proof}
	
	
	\begin{proof}[Proof of corollary \ref{corollary1}][Triangle inequality of diffusion distance]
	\end{proof}
	
	Let $x, y, z \in V(G).$
	
	\begin{equation}
	\begin{split}
	D^{2}_{t}(x,z) & = \sum\limits_{w \in V(G)} \big( P^{t}(x,w) - P^{t}(z,w)   \big)^2 \frac{1}{\pi(w)}  \\ & = \sum\limits_{w \in V(G)} \big(P^{t}(x, w) - P^{t}(y,w) + P^{t}(y,w) - P^{t}(z,w) \big)^2 \frac{1}{\pi(w)} \\ & = \sum\limits_{w \in V(G)} \big( P^{t}(x,w) - P^{t}(y,w) \big)^2 \frac{1}{\pi(w)}  + \sum\limits_{w \in V(G)} \big( P^{t}(y,w) - P^{t}(z,w)  \big)^2 \frac{1}{\pi(w)} \\ & + 2 \sum\limits_{w \in V(G)} \big( P^{t}(x,w) - P^{t}(y,w)  \big) \big( P^{t}(y,w) - P^{t}(z,w)  \big)\frac{1}{\pi(w)} \\ &= D^{2}_{t}(x,y) + D^{2}_{t}(y,z) +  2 \sum\limits_{w \in V(G)} \big( P^{t}(x,w) - P^{t}(y,w)  \big) \big( P^{t}(y,w) - P^{t}(z,w)  \big)\frac{1}{\pi(w)}   
	\end{split}
	\end{equation}
	
	
	Thus it suffices to show that 
	
	\begin{equation}
	\sum\limits_{w \in V(G)} \big( P^{t}(x,w) - P^{t}(y,w)  \big) \big( P^{t}(y,w) - P^{t}(z,w)  \big)\frac{1}{\pi(w)} \leq D_{t}(x,y) \cdot D_{t}(y,z). 
	\end{equation}
	
	Let $a_{w} = \big(P^{t}(x,w) - P^{t}(y,w) \big) \sqrt{1 / \pi(w)}$ and $b_{w} = \big( P^{t}(y,w) - P^{t}(z,w) \big) \sqrt{1 / \pi(w)}$. Then the above inequality is equivalent to :
	
	\begin{equation} 
	\sum\limits_{w \in V(G)} a_{w} \cdot b_{w} \leq \sqrt{\sum\limits_{w \in V(G)} a^2_{w} \cdot \sum\limits_{w \in V(G)} b^2_{w} }.
	\end{equation}
	
	,which is true by Cauchy-Schwarz inequality.
	
	
	
	\section{BibTeX}
	
	We hope you've chosen to use BibTeX!\ If you have, please feel free to use the package natbib with any bibliography style you're comfortable with. The .bst file agsm has been included here for your convenience. 
	
	\bibliographystyle{Chicago}
	\bibliography{Biblio}
	
	
	\newpage
	\bigskip
	\begin{center}
		{\large\bf SUPPLEMENTARY MATERIAL}
	\end{center}
	
	\begin{description}
		
		\item[Title:] Brief description. (file type)
		
		\item[R-package for  MYNEW routine:] R-package ÒMYNEWÓ containing code to perform the diagnostic methods described in the article. The package also contains all datasets used as examples in the article. (GNU zipped tar file)
		
		\item[HIV data set:] Data set used in the illustration of MYNEW method in Section~ 3.2. (.txt file)
	\end{description}
	
	
\end{document}

%%%%%%%%%%%%%%%%%%%%%%%%%%%

